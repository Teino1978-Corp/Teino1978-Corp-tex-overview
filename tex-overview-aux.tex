%% This document is part of the package tex-overview and is only useable with the main file tex-overview.tex.
%% author: Arno Trautmann

% !TeX root = tex-overview.tex
%% (for the TeXworks-users.)

\usepackage{
  bookmark,
  cooltooltips,
  dtklogos,  %% for \NTS etc.
  fontspec,
  geometry,
  hyperref,
  pdftexcmds,
  tikz,
  xcolor,
  xparse
}
%% circumventing a bug in cooltooltips
\makeatletter
  \let\topdfescapestring\pdf@escapestring
\makeatother

\ExplSyntaxNamesOn


%% define the look-and-feel of the document
\setmainfont{TeX Gyre Pagella}
\setsansfont{TeX Gyre Pagella}


%% setup the style of hyperlinks.
\hypersetup{
  colorlinks=true,
  linkcolor=blue,
  urlcolor=blue,
  pdfborder=0 0 0   %% don’t like these boxes
}

%% page layout, headings
\pagestyle{empty}
\addtokomafont{disposition}{\color[rgb]{.4 0 0}}
\addtokomafont{section}{\Huge}
\addtokomafont{subsection}{\LARGE}
\addtokomafont{subsubsection}{\Large}

%% some missing definitions for \TeX-like things
\tl_set:Nn\XeT{X\kern -.1667em\lower .5ex\hbox {E}\kern -.125emT\@}
\tl_set:Nn\ConTeXt{Con\TeX{}t}

%% constants for the colors. Might change from time to time …
\tl_set:Nn\vip{red}
\tl_set:Nn\normalimportant{blue}
\tl_set:Nn\experimental{yellow}
\tl_set:Nn\fonttechnology{pink}
\tl_set:Nn\program{black}
\tl_set:Nn\package{green}
\tl_set:Nn\distro{orange}

%% shorthand to keep a good structure of the node positions
\fp_new:N\layer
\fp_new:N\layerdist
\fp_set:Nn\layer{0}
\fp_set:Nn\layerdist{-1}

%% width of the tooltip-boxes (negative value to make them disappear at all)
\dim_set:Nn\fboxrule{-1mm}

%% to separate important from not-so-important nodes
\bool_new:N\to_short

%% the token list to save all the textviews. When set once, it is reset. May be useful anyhow
\tl_set:Nn\to_textviews{}
\NewDocumentCommand\settextviews{}{
  \to_textviews
  \tl_set:Nn\to_textviews{}
}
\cs_gset:Nn\addtotextviews:f{\tl_gput_right:No\to_textviews{#1}}

%% environment to set the graphs
\NewDocumentCommand\tograph{sD(){}+m}
{
  \IfBooleanT{#1}{
    \ExplSyntaxNamesOn
    \bool_gset_true:N\to_short
    \ExplSyntaxNamesOff
    #2
    \begin{tikzpicture}
      #3
    \end{tikzpicture}
    \clearpage
  }
  \ExplSyntaxNamesOn
  \bool_gset_false:N\to_short
  \ExplSyntaxNamesOff
  #2
  \begin{tikzpicture}
  	#3
  \end{tikzpicture}
}

%% Style of the nodes: shade from a light 
\tikzstyle{coolnode} = [
	draw=\nodecolor!50!black!70,
	top color=\nodecolor!10!white!105,
	bottom color= \nodecolor!50!black!50
]

%% set the first default node style (will change to \distro or \program in the document)
\cs_set_eq:NN\tonodestyle\normalimportant

%% tonode ⇒ tex-overview node, now with better argument specifier
\NewDocumentCommand\tonode{O{\tonodestyle}D(){no label given}D(){no position given}D<>{no description given}m}{
%% save the content pdfescaped
  \tl_set:No\tonodecontent{\topdfescapestring{#4}}

%% test if we are in the short or full view
  \tl_gset:Nn\nodecolor{#1}
  \bool_if:NTF\to_short{
  %% short view
    \tl_if_eq:NNT#1{\vip}{   %% then check if this is an important node
      \node[coolnode]
	(#2) at (#3) {
        \cooltooltip{#2}{\tonodecontent}{\tonodecontent}{}{#5\strut}
      };
    }
  }{
  %% long view
    \node[coolnode] (#2) at (#3) {
      \cooltooltip{#2}{\tonodecontent}{\tonodecontent}{}{#5\strut}
    };
  %% and the text view, to be added only once!
  \addtotextviews:f{\subsubsection*{\color{blue}#5}\parbox{\columnwidth}{#4}}
  }
}

%% we want to make use of two pdf layers: the upper (main) one for the nodes
%% and the lower (background) one for the lines. That way, the lines will not cross the nodes
\pgfdeclarelayer{background}
\pgfsetlayers{background,main}

%% the command to draw from one node to the other one. Fine tuning is possible via optional argument #3
\NewDocumentCommand{\todraw}{st{'}t{.}t{-}O{}d()d()}{
  \begin{pgfonlayer}{background}
    %% draw in the case that: either a * is given (always draw) or (' is given and \to_short) or nothing is given
    \bool_if:nT{#1 || (#2 && \to_short) || !(#2 || \to_short)}
      {\draw [thick,gray,\IfBooleanT{#3}{dotted},\IfBooleanT{#4}{dashed},#5] (#6) to (#7);}  %% I’m just loooving expl3!
  \end{pgfonlayer}
}

%% a nice way to control the vertical position of nodes
\NewDocumentCommand{\setlayer}{m}{
  \fp_set:Nn\layer{#1}
}
\NewDocumentCommand{\steplayer}{O{\layerdist}}{
  \fp_add:Nn\layer{#1}
}

\NewDocumentCommand\topart{m}{
  \dim_set:Nn\pdfpagewidth{9cm}       %% make a smaller paper so the header won’t feel so alone on the big, cold paper
  \dim_set:Nn\pdfpageheight{4cm}
	\newgeometry{margin=1cm}  %% to save space; no need for margins if only a tree is shown
	\part{#1}
	\newpage
	\restoregeometry
  \dim_set:Nn\pdfpagewidth{21cm}     %% back to a4
  \dim_set:Nn\pdfpageheight{29.7cm}
}

%% the following code is made to avoid code doubling on cost of readability. It is more an experiment of mine. This code is neither stable nor exemplary …
%% the first argument, given in (), determines the level of the d
\NewDocumentCommand\todisposition{D(){0}omO{\tonodestyle}}{
  \tl_set:Nn\to_disp{section}
  \int_compare:nT{#1 > 0}{\tl_put_left:Nn\to_disp{sub}}
  \int_compare:nT{#1 > 1}{\tl_put_left:Nn\to_disp{sub}}
  \bool_if:NTF\to_short
  {
    \int_compare:nT{#1 = 0}{\stepcounter{section}}
    \tl_gset:cn{to_mand\to_disp}{\cs:w the\to_disp\cs_end:.\hspace{.175em} #3\newline short view}
    \IfNoValueTF{#2}{\tl_gset:cn{to_opt\to_disp sect}{\tl_use:c{to_mand\to_disp}}}{\tl_gset:cn{to_opt\to_disp}{#2, short view}}
    \cs:w \to_disp \cs_end:*{\cs:w to_mand\to_disp \cs_end:}
    \int_compare:nT{#1 = 0}{\addtocounter{section}{-1}}
  }
  {
    \tl_gset:cn{to_mand\to_disp}{#3}
    \IfNoValueTF{#2}{\tl_gset:cn{to_opt\to_disp}{\cs:w to_mand\to_disp\cs_end:}}{\tl_gset:cn{to_opt\to_disp}{#2}}
    \addtotextviews:f{\cs:w \to_disp\cs_end:[#2]{#3}}
    \cs:w \to_disp\cs_end:[\cs:w to_opt\to_disp\cs_end:]{\cs:w to_mand\to_disp\cs_end:}
  }
  \cs_gset_eq:NN\tonodestyle#4
}

\NewDocumentCommand\overviewsection{omO{\tonodestyle}}{
  \bool_if:NTF\to_short
  {
    \stepcounter{section}
    \tl_gset:Nn\to_mandsect{\thesection.\hspace{.2em} #2\newline short view}
    \IfNoValueTF{#1}{\tl_gset:Nn\to_optsect{\to_mandsect}}{\tl_gset:Nn\to_optsect{#1, short view}}
    \section*{\to_mandsect}
    \addtocounter{section}{-1}
  }
  {
    \tl_gset:Nn\to_mandsect{#2}
    \IfNoValueTF{#1}{\tl_gset:Nn\to_optsect{\to_mandsect}}{\tl_gset:Nn\to_optsect{#1}}
    \addtotextviews:f{\section[#1]{#2}}
    \section[\to_optsect]{\to_mandsect}
  }
  \cs_gset_eq:NN\tonodestyle#3
}

\NewDocumentCommand{\overviewsubsection}{omO{\tonodestyle}}{
  \tl_set:Nn\to_mandsubsect{#2}
  \bool_if:NTF\to_short
  {
    \IfNoValueTF{#1}{\tl_set:Nn\to_optsubsect{\to_mandsubsect}}{\tl_set:Nn\to_optsubsect{#1, short view}}
    \subsection*{\to_mandsubsect}
  }
  {
    \IfNoValueTF{#1}{\tl_set:Nn\to_optsubsect{\to_mandsubsect}}{\tl_set:Nn\to_optsubsect{#1}}
    \addtotextviews:f{\subsection[#1]{#2}} %% there should be \to_opt[]sec in the first arg, but I’m too dumb to get it right
    \subsection[\to_optsubsect]{\to_mandsubsect}
  }
  \cs_gset_eq:NN\tonodestyle#3
}

\NewDocumentCommand{\overviewsubsubsection}{omO{\tonodestyle}}{
  \tl_set:Nn\to_mandsubsubsect{#2}
  \bool_if:NTF\to_short
  {
    \IfNoValueTF{#1}{\tl_set:Nn\to_optsubsubsect{\to_mandsubsubsect}}{\tl_set:Nn\to_optsubsubsect{#1, short view}}
  }
  {
    \IfNoValueTF{#1}{\tl_set:Nn\to_optsubsubsect{\to_mandsubsubsect}}{\tl_set:Nn\to_optsubsubsect{#1}}
    \addtotextviews:f{\subsubsection[#1]{#2}}  %% there should be \to_opt[]sec in the first arg, but I’m too dumb to get it right
  }

  \subsubsection[\to_optsubsubsect]{\to_mandsubsubsect}
  \cs_gset_eq:NN\tonodestyle#3
}

%% a command for creation of bib-items.
\NewDocumentCommand\tobibsection{m}{
  \subsection*{#1}
}
\NewDocumentCommand\tobibsubsection{m}{
  \subsection*{#1}
}
\NewDocumentCommand\tobib{mD<>{}O{}}{
  \vspace*{1.5ex}
\begin{minipage}{\textwidth}  %% to prevent page breaks
  \large #1\normalsize\\ \hspace*{1em} \parbox{\textwidth-1.5em}{\url{#2} #3}
\end{minipage}
}

\ExplSyntaxNamesOff

\AtBeginDocument{
  {
    \centering
    \huge\bfseries An overview of \TeX, its children\\ and their friends~\dots\par
    \flushright\parbox{4cm}{
    \Large \color[rgb]{.4 0 0} Arno Trautmann\\ \fontsize{9.85}{10}\selectfont arno.trautmann@gmx.de}
	\hspace*{2cm}\par
  }
	\vspace*{1cm}
}